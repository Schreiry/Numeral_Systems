\documentclass[12pt,a4paper]{article}

% Кодировка и поддержка английского языка
\usepackage[utf8]{inputenc}
\usepackage[T1]{fontenc}
\usepackage[english]{babel}

% Пакеты для графики, математики и оформления
\usepackage{graphicx}
\usepackage{amsmath, amssymb}
\usepackage{enumitem}
\usepackage{hyperref}
\usepackage{float}

% Поля страницы
\usepackage[a4paper, top=2cm, bottom=2cm, left=2cm, right=2cm]{geometry}

% Колонтитулы
\usepackage{fancyhdr}
\pagestyle{fancy}
\fancyhf{}
\rhead{\thepage}
\lhead{Numeral Systems}

% Интервал между абзацами
\setlength{\parskip}{1em}

% Заголовок документа
\title{\textbf{A Comprehensive Guide to Numeral Systems:\\ Binary, Hexadecimal, Octal, and Decimal}}
\author{Student: David Greve.\\Faculty of Informatics}
\date{\today}

\begin{document}

% Title Page
\begin{titlepage}
    \centering
    \vspace*{1cm}
    \includegraphics[width=0.3\textwidth]{Georgian-Tech-logo.png}\\[1cm]
    {\LARGE \textbf{A Comprehensive Guide to Numeral Systems}}\\[1cm]
    {\Large Binary, Hexadecimal, Octal, and Decimal}\\[2cm]
    \textbf{Student:} David Greve.\\[0.5cm]
    \textbf{Instructor:} Vasil Kuciava.\\[2cm]
    {\large \today}
    \vfill
\end{titlepage}

% Table of Contents
\tableofcontents
\newpage

%%%%%%%%%%%%%%%%%%%%%%%%%%%%%%%%%%%%%%%%%%%%%%%%%%%%%%%%%%%%%%
\section{Introduction}
This document discusses the basic numeral systems: Binary (base 2), Octal (base 8), Decimal (base 10), and Hexadecimal (base 16). Detailed methods for converting numbers between these systems are provided along with numerous examples and solved exercises. The goal is to offer a deep understanding of these numeral systems and demonstrate practical conversion techniques.

%%%%%%%%%%%%%%%%%%%%%%%%%%%%%%%%%%%%%%%%%%%%%%%%%%%%%%%%%%%%%%
\section{General Concepts of Numeral Systems}
A numeral system is a method for representing numbers using a set of symbols (digits) and a specific base \(b\). Any number can be represented in base \(b\) as:
\[
N = a_n b^n + a_{n-1} b^{n-1} + \dots + a_1 b + a_0,
\]
where \(0 \leq a_i < b\). The primary numeral systems discussed in this document are:
\begin{itemize}[label=\(\bullet\)]
    \item \textbf{Binary:} \(b = 2\) (digits: 0, 1);
    \item \textbf{Octal:} \(b = 8\) (digits: 0, 1, \dots, 7);
    \item \textbf{Decimal:} \(b = 10\) (digits: 0, 1, \dots, 9);
    \item \textbf{Hexadecimal:} \(b = 16\) (digits: 0, 1, \dots, 9, A, B, C, D, E, F).
\end{itemize}

%%%%%%%%%%%%%%%%%%%%%%%%%%%%%%%%%%%%%%%%%%%%%%%%%%%%%%%%%%%%%%
\section{The Binary System}
The binary system is based on two digits: 0 and 1. Each digit represents a power of 2. For example, the number:
\[
1011_2 = 1 \cdot 2^3 + 0 \cdot 2^2 + 1 \cdot 2^1 + 1 \cdot 2^0 = 8 + 0 + 2 + 1 = 11_{10}.
\]

\subsection{Conversion from Binary to Decimal}
To convert a binary number to decimal, multiply each digit by the corresponding power of 2 and sum the results:
\[
\text{If } N = d_n d_{n-1}\dots d_0, \text{ then } N_{10} = \sum_{i=0}^{n} d_i \cdot 2^i.
\]
\textbf{Example:} Convert \(1101_2\) to decimal.
\[
1101_2 = 1\cdot2^3 + 1\cdot2^2 + 0\cdot2^1 + 1\cdot2^0 = 8 + 4 + 0 + 1 = 13_{10}.
\]

%%%%%%%%%%%%%%%%%%%%%%%%%%%%%%%%%%%%%%%%%%%%%%%%%%%%%%%%%%%%%%
\section{The Octal System}
The octal system uses digits from 0 to 7. Each digit corresponds to a power of 8:
\[
N = a_n 8^n + a_{n-1} 8^{n-1} + \dots + a_0.
\]

\subsection{Conversion from Octal to Decimal}
\textbf{Example:} Convert \(725_8\) to decimal.
\[
725_8 = 7\cdot8^2 + 2\cdot8^1 + 5\cdot8^0 = 7\cdot64 + 2\cdot8 + 5 = 448 + 16 + 5 = 469_{10}.
\]

%%%%%%%%%%%%%%%%%%%%%%%%%%%%%%%%%%%%%%%%%%%%%%%%%%%%%%%%%%%%%%
\section{The Decimal System}
The decimal system is the standard numeral system, with base 10. It uses the digits 0 through 9.

\subsection{Conversion from Decimal to Other Systems}
To convert a decimal number to another numeral system, perform successive division by the target base. The remainders, read in reverse order, form the converted number.

\textbf{Example:} Convert \(345_{10}\) to binary.
\begin{enumerate}
    \item \(345 \div 2 = 172\) with a remainder of \(1\).
    \item \(172 \div 2 = 86\) with a remainder of \(0\).
    \item \(86 \div 2 = 43\) with a remainder of \(0\).
    \item \(43 \div 2 = 21\) with a remainder of \(1\).
    \item \(21 \div 2 = 10\) with a remainder of \(1\).
    \item \(10 \div 2 = 5\) with a remainder of \(0\).
    \item \(5 \div 2 = 2\) with a remainder of \(1\).
    \item \(2 \div 2 = 1\) with a remainder of \(0\).
    \item \(1 \div 2 = 0\) with a remainder of \(1\).
\end{enumerate}
Reading the remainders in reverse order gives:
\[
345_{10} = 101011001_2.
\]

%%%%%%%%%%%%%%%%%%%%%%%%%%%%%%%%%%%%%%%%%%%%%%%%%%%%%%%%%%%%%%
\section{The Hexadecimal System}
The hexadecimal system is based on 16. In addition to the digits 0--9, the letters A, B, C, D, E, and F are used to represent values 10 through 15.

\subsection{Conversion from Hexadecimal to Decimal}
\textbf{Example:} Convert \(1A3_{16}\) to decimal.
\[
1A3_{16} = 1\cdot16^2 + 10\cdot16^1 + 3\cdot16^0 = 256 + 160 + 3 = 419_{10}.
\]

%%%%%%%%%%%%%%%%%%%%%%%%%%%%%%%%%%%%%%%%%%%%%%%%%%%%%%%%%%%%%%
\section{Conversion Methods Between Numeral Systems}
This section describes the main algorithms for converting numbers between different numeral systems.

\subsection{Binary \(\leftrightarrow\) Decimal}
\begin{itemize}
    \item \textbf{Binary to Decimal:} Multiply each binary digit by the corresponding power of 2 and sum the results.
    \item \textbf{Decimal to Binary:} Use successive division by 2 and record the remainders.
\end{itemize}

\subsection{Binary \(\leftrightarrow\) Octal}
Conversion between binary and octal is simplified because \(8 = 2^3\). To convert:
\begin{enumerate}
    \item Group the binary number into sets of 3 bits starting from the right.
    \item Replace each group with the corresponding octal digit.
\end{enumerate}

\textbf{Example:} Convert \(1011101_2\) to octal.
\[
1011101_2 \quad \Rightarrow \quad \underline{1}\, 011\, 101.
\]
Add leading zeros if necessary (e.g., \(001\)) to form groups: \(001\), \(011\), \(101\).
\begin{itemize}
    \item \(001_2 = 1_8\),
    \item \(011_2 = 3_8\),
    \item \(101_2 = 5_8\).
\end{itemize}
Thus, \(1011101_2 = 135_8\).

\subsection{Binary \(\leftrightarrow\) Hexadecimal}
Since \(16 = 2^4\), the process is analogous:
\begin{enumerate}
    \item Group the binary number into sets of 4 bits starting from the right.
    \item Replace each group with the corresponding hexadecimal digit.
\end{enumerate}

\textbf{Example:} Convert \(10101100_2\) to hexadecimal.
\[
10101100_2 \quad \Rightarrow \quad 1010 \quad 1100.
\]
\begin{itemize}
    \item \(1010_2 = A_{16}\) (since \(10_{10}=A\)),
    \item \(1100_2 = C_{16}\) (since \(12_{10}=C\)).
\end{itemize}
Thus, \(10101100_2 = AC_{16}\).

\subsection{Hexadecimal \(\leftrightarrow\) Octal}
Conversion between hexadecimal and octal is often performed via binary:
\begin{enumerate}
    \item Convert the hexadecimal number to binary (by replacing each hexadecimal digit with 4 bits).
    \item Convert the resulting binary number to octal by grouping the bits into sets of 3.
\end{enumerate}

%%%%%%%%%%%%%%%%%%%%%%%%%%%%%%%%%%%%%%%%%%%%%%%%%%%%%%%%%%%%%%
\section{Detailed Examples and Exercises}

\subsection{Exercise 1: Convert from Binary to Hexadecimal}
\textbf{Problem:} Convert \(11010110_2\) to hexadecimal.

\textbf{Solution:}
\begin{enumerate}
    \item Group the binary number into sets of 4 bits starting from the right:
    \[
    11010110_2 \quad \Rightarrow \quad 1101 \quad 0110.
    \]
    \item Convert each group:
    \begin{itemize}
        \item \(1101_2 = 1\cdot2^3+1\cdot2^2+0\cdot2^1+1\cdot2^0 = 8+4+0+1 = 13 \Rightarrow D_{16}\),
        \item \(0110_2 = 0\cdot2^3+1\cdot2^2+1\cdot2^1+0\cdot2^0 = 0+4+2+0 = 6 \Rightarrow 6_{16}\).
    \end{itemize}
    \item Therefore, \(11010110_2 = D6_{16}\).
\end{enumerate}

\subsection{Exercise 2: Convert from Hexadecimal to Binary}
\textbf{Problem:} Convert \(2F4_{16}\) to binary.

\textbf{Solution:}
\begin{enumerate}
    \item Convert each hexadecimal digit to a 4-bit binary number:
    \begin{itemize}
        \item \(2_{16} = 0010_2\),
        \item \(F_{16} = 15_{10} = 1111_2\),
        \item \(4_{16} = 0100_2\).
    \end{itemize}
    \item Combine the groups:
    \[
    2F4_{16} = 0010\,1111\,0100_2.
    \]
    \item The final binary number is \(001011110100_2\) (leading zeros may be omitted).
\end{enumerate}

\subsection{Exercise 3: Convert from Octal to Decimal}
\textbf{Problem:} Convert \(657_8\) to decimal.

\textbf{Solution:}
\[
657_8 = 6\cdot8^2 + 5\cdot8^1 + 7\cdot8^0 = 6\cdot64 + 5\cdot8 + 7 = 384 + 40 + 7 = 431_{10}.
\]

\subsection{Exercise 4: Convert from Decimal to Binary}
\textbf{Problem:} Convert \(237_{10}\) to binary.

\textbf{Solution:} Use successive division by 2:
\begin{enumerate}
    \item \(237 \div 2 = 118\) with remainder \(1\).
    \item \(118 \div 2 = 59\) with remainder \(0\).
    \item \(59 \div 2 = 29\) with remainder \(1\).
    \item \(29 \div 2 = 14\) with remainder \(1\).
    \item \(14 \div 2 = 7\) with remainder \(0\).
    \item \(7 \div 2 = 3\) with remainder \(1\).
    \item \(3 \div 2 = 1\) with remainder \(1\).
    \item \(1 \div 2 = 0\) with remainder \(1\).
\end{enumerate}
Reading the remainders in reverse order: \(1\,1\,1\,0\,1\,1\,0\,1\). Thus,
\[
237_{10} = 11101101_2.
\]

\subsection{Exercise 5: Convert from Binary to Octal}
\textbf{Problem:} Convert \(101110111_2\) to octal.

\textbf{Solution:}
\begin{enumerate}
    \item Group the binary number into sets of 3 bits starting from the right:
    \[
    101110111_2 \quad \Rightarrow \quad 101\,110\,111.
    \]
    \item Convert each group:
    \begin{itemize}
        \item \(101_2 = 5_8\),
        \item \(110_2 = 6_8\),
        \item \(111_2 = 7_8\).
    \end{itemize}
    \item Hence, \(101110111_2 = 567_8\).
\end{enumerate}

\subsection{Exercise 6: Convert from Hexadecimal to Octal}
\textbf{Problem:} Convert \(3B_{16}\) to octal.

\textbf{Solution:}
\begin{enumerate}
    \item First, convert the hexadecimal number to binary by replacing each hexadecimal digit with 4 bits:
    \begin{itemize}
        \item \(3_{16} = 0011_2\),
        \item \(B_{16} = 11_{10} = 1011_2\) (pad with zeros if necessary).
    \end{itemize}
    \item Combine the binary digits: \(3B_{16} = 0011\,1011_2\).
    \item Group the binary number into sets of 3 bits starting from the right:
    \[
    0011\,1011_2 \quad \Rightarrow \quad 00\, 111\, 011.
    \]
    Add leading zeros to the first group if needed.
    \item Convert each group:
    \begin{itemize}
        \item \(000_2 = 0_8\),
        \item \(111_2 = 7_8\),
        \item \(011_2 = 3_8\).
    \end{itemize}
    \item Therefore, \(3B_{16} = 073_8\) (leading zero may be omitted, yielding \(73_8\)).
\end{enumerate}

%%%%%%%%%%%%%%%%%%%%%%%%%%%%%%%%%%%%%%%%%%%%%%%%%%%%%%%%%%%%%%
\section{Notes and Tips}
\begin{itemize}
    \item When converting from binary to octal or hexadecimal, be sure to group the bits correctly (3 bits for octal, 4 bits for hexadecimal). If the total number of bits is not divisible by 3 or 4, add leading zeros.
    \item To verify the conversion, try converting back to the original numeral system.
    \item In numeral systems with a base greater than 10, remember the correspondence between digits and letters (e.g., \(A=10,\, B=11,\, \dots, F=15\)).
\end{itemize}

%%%%%%%%%%%%%%%%%%%%%%%%%%%%%%%%%%%%%%%%%%%%%%%%%%%%%%%%%%%%%%
\section{Conclusion}
This document has explored the fundamentals of various numeral systems and provided methods for converting numbers between binary, octal, decimal, and hexadecimal systems. Detailed examples and exercises help reinforce the theoretical knowledge with practical application. Mastery of these conversion techniques is essential for students in computer science and programming, as these numeral systems are widely used in computing and algorithms.

%%%%%%%%%%%%%%%%%%%%%%%%%%%%%%%%%%%%%%%%%%%%%%%%%%%%%%%%%%%%%%
\section*{References}
\addcontentsline{toc}{section}{References}
\begin{enumerate}
    \item Kulikov, A.V. "Fundamentals of Discrete Mathematics."
    \item Levin, A. "Computer Science: Theory and Practice."
    \item Methodological materials from the Department of Informatics.
\end{enumerate}

%%%%%%%%%%%%%%%%%%%%%%%%%%%%%%%%%%%%%%%%%%%%%%%%%%%%%%%%%%%%%%
\section*{Appendix: Additional Examples}
\addcontentsline{toc}{section}{Appendix: Additional Examples}

\textbf{Example A.} Convert \(1001110_2\) to hexadecimal.
\begin{enumerate}
    \item Group into 4-bit groups: \(0100\ 1110_2\).
    \item \(0100_2 = 4_{16}\), \(1110_2 = E_{16}\).
    \item Answer: \(1001110_2 = 4E_{16}\).
\end{enumerate}

\textbf{Example B.} Convert \(572_{10}\) to octal.
\begin{enumerate}
    \item Divide by 8:
    \begin{itemize}
        \item \(572 \div 8 = 71\) with remainder \(4\).
        \item \(71 \div 8 = 8\) with remainder \(7\).
        \item \(8 \div 8 = 1\) with remainder \(0\).
        \item \(1 \div 8 = 0\) with remainder \(1\).
    \end{itemize}
    \item Remainders read in reverse order: \(1\,0\,7\,4\).
    \item Answer: \(572_{10} = 1074_8\).
\end{enumerate}

\textbf{Example C.} Convert \(3C9_{16}\) to decimal.
\[
3C9_{16} = 3\cdot16^2 + 12\cdot16^1 + 9\cdot16^0 = 3\cdot256 + 192 + 9 = 768 + 192 + 9 = 969_{10}.
\]

\vfill
\textbf{Note:} This material is intended as a comprehensive study resource for students learning the fundamentals of numeral systems.

\end{document}
