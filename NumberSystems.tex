\documentclass[12pt,a4paper]{article}

% Кодировка и поддержка русского языка
\usepackage[T2A]{fontenc}
\usepackage[utf8]{inputenc}
\usepackage[russian]{babel}

% Пакеты для графики и математических формул
\usepackage{graphicx}
\usepackage{amsmath, amssymb}
\usepackage{enumitem}
\usepackage{hyperref}
\usepackage{float}

% Поля страницы
\usepackage[a4paper, top=2cm, bottom=2cm, left=2cm, right=2cm]{geometry}

% Колонтитулы
\usepackage{fancyhdr}
\pagestyle{fancy}
\fancyhf{}
\rhead{\thepage}
\lhead{Системы счисления}

% Интервал между абзацами
\setlength{\parskip}{1em}

% Заголовок документа
\title{\textbf{Полное руководство по системам счисления: \\ Бинарная, Шестнадцатеричная, Восьмеричная и Десятичная}}
\author{Студент: Греве Давид.\\Факультет Информатики}
\date{\today}

\begin{document}

% Титульная страница
\begin{titlepage}
    \centering
    % Фотография (файл photo.jpg должен находиться в одной директории с документом)
    \vspace*{1cm}
    \includegraphics[width=0.3\textwidth]{Georgian-Tech-logo.png}\\[1cm]
    {\LARGE \textbf{Полное руководство по системам счисления}}\\[1cm]
    {\Large Бинарная, шестнадцатеричная, восьмеричная и десятичная системы}\\[2cm]
    \textbf{Студент:} Греве Давид.\\[0.5cm]
    \textbf{Преподаватель:} Куциава Василий.\\[2cm]
    {\large \today}
    \vfill
\end{titlepage}

% Оглавление
\tableofcontents
\newpage

%%%%%%%%%%%%%%%%%%%%%%%%%%%%%%%%%%%%%%%%%%%%%%%%%%%%%%%%%%%%%%
\section{Введение}
В данной работе рассматриваются основные системы счисления: бинарная (основание 2), восьмеричная (основание 8), десятичная (основание 10) и шестнадцатеричная (основание 16). Пошагово разобраны методы преобразования чисел между указанными системами, приведены примеры и задачи с подробным решением. Цель работы --- дать читателю глубокое понимание устройства этих систем, а также показать на практике, как выполнять их преобразования.

%%%%%%%%%%%%%%%%%%%%%%%%%%%%%%%%%%%%%%%%%%%%%%%%%%%%%%%%%%%%%%
\section{Общие понятия о системах счисления}
Система счисления --- способ записи чисел с использованием заданного набора символов (цифр) и определённой базы (основания). Любое число можно представить в системе с основанием \(b\) в виде:
\[
N = a_n b^n + a_{n-1} b^{n-1} + \dots + a_1 b + a_0,
\]
где \(0 \leq a_i < b\). Основные системы, рассматриваемые в данной работе:
\begin{itemize}[label=\(\bullet\)]
    \item \textbf{Бинарная:} \(b = 2\) (цифры: 0, 1);
    \item \textbf{Восьмеричная:} \(b = 8\) (цифры: 0, 1, \dots, 7);
    \item \textbf{Десятичная:} \(b = 10\) (цифры: 0, 1, \dots, 9);
    \item \textbf{Шестнадцатеричная:} \(b = 16\) (цифры: 0, 1, \dots, 9, A, B, C, D, E, F).
\end{itemize}

%%%%%%%%%%%%%%%%%%%%%%%%%%%%%%%%%%%%%%%%%%%%%%%%%%%%%%%%%%%%%%
\section{Бинарная система счисления}
Бинарная система основана на двух цифрах: 0 и 1. Каждый разряд числа соответствует степени двойки. Например, число:
\[
1011_2 = 1 \cdot 2^3 + 0 \cdot 2^2 + 1 \cdot 2^1 + 1 \cdot 2^0 = 8 + 0 + 2 + 1 = 11_{10}.
\]

\subsection{Преобразование из двоичной в десятичную систему}
Чтобы перевести число из бинарной системы в десятичную, необходимо перемножить каждую цифру на соответствующую степень двойки и просуммировать:
\[
\text{Если } N = d_n d_{n-1}\dots d_0, \text{ то } N_{10} = \sum_{i=0}^{n} d_i \cdot 2^i.
\]
\textbf{Пример:} Переведём \(1101_2\) в десятичную систему.
\[
1101_2 = 1\cdot2^3 + 1\cdot2^2 + 0\cdot2^1 + 1\cdot2^0 = 8 + 4 + 0 + 1 = 13_{10}.
\]

%%%%%%%%%%%%%%%%%%%%%%%%%%%%%%%%%%%%%%%%%%%%%%%%%%%%%%%%%%%%%%
\section{Восьмеричная система счисления}
Восьмеричная система использует цифры от 0 до 7. Каждая цифра соответствует степени восьмерки.
\[
N = a_n 8^n + a_{n-1} 8^{n-1} + \dots + a_0.
\]

\subsection{Преобразование из восьмеричной в десятичную систему}
Пример: Переведём число \(725_8\) в десятичную систему.
\[
725_8 = 7\cdot8^2 + 2\cdot8^1 + 5\cdot8^0 = 7\cdot64 + 2\cdot8 + 5 = 448 + 16 + 5 = 469_{10}.
\]

%%%%%%%%%%%%%%%%%%%%%%%%%%%%%%%%%%%%%%%%%%%%%%%%%%%%%%%%%%%%%%
\section{Десятичная система счисления}
Десятичная система --- это стандартная система, основание которой равно 10. В ней используются цифры от 0 до 9.

\subsection{Преобразование из десятичной в другие системы}
Преобразование числа из десятичной системы в любую другую выполняется с помощью последовательного деления на основание целевой системы. Остатки от деления, записанные в обратном порядке, дают искомое число.

\textbf{Пример:} Переведём \(345_{10}\) в двоичную систему.
\begin{enumerate}
    \item \(345 \div 2 = 172\) с остатком \(1\).
    \item \(172 \div 2 = 86\) с остатком \(0\).
    \item \(86 \div 2 = 43\) с остатком \(0\).
    \item \(43 \div 2 = 21\) с остатком \(1\).
    \item \(21 \div 2 = 10\) с остатком \(1\).
    \item \(10 \div 2 = 5\) с остатком \(0\).
    \item \(5 \div 2 = 2\) с остатком \(1\).
    \item \(2 \div 2 = 1\) с остатком \(0\).
    \item \(1 \div 2 = 0\) с остатком \(1\).
\end{enumerate}
Записывая остатки в обратном порядке, получаем:
\[
345_{10} = 101011001_2.
\]

%%%%%%%%%%%%%%%%%%%%%%%%%%%%%%%%%%%%%%%%%%%%%%%%%%%%%%%%%%%%%%
\section{Шестнадцатеричная система счисления}
Шестнадцатеричная система основана на числе 16. Помимо цифр 0--9 используются буквы A, B, C, D, E, F, обозначающие числа 10, 11, 12, 13, 14, 15 соответственно.

\subsection{Преобразование из шестнадцатеричной в десятичную систему}
\textbf{Пример:} Переведём \(1A3_{16}\) в десятичную систему.
\[
1A3_{16} = 1\cdot16^2 + 10\cdot16^1 + 3\cdot16^0 = 256 + 160 + 3 = 419_{10}.
\]

%%%%%%%%%%%%%%%%%%%%%%%%%%%%%%%%%%%%%%%%%%%%%%%%%%%%%%%%%%%%%%
\section{Методы преобразования между системами счисления}
В данном разделе описаны основные алгоритмы перевода чисел между системами счисления.

\subsection{Бинарная \(\leftrightarrow\) Десятичная}
\begin{itemize}
    \item \textbf{Из двоичной в десятичную:} Как уже описывалось, умножаем каждую цифру на соответствующую степень 2 и суммируем.
    \item \textbf{Из десятичной в двоичную:} Последовательное деление числа на 2 с записью остатков.
\end{itemize}

\subsection{Двоичная \(\leftrightarrow\) Восьмеричная}
Преобразование между двоичной и восьмеричной системами упрощается за счёт того, что \(8 = 2^3\). Для преобразования:
\begin{enumerate}
    \item Разбиваем двоичное число на группы по 3 бита, начиная с младших разрядов.
    \item Каждую группу заменяем соответствующей восьмеричной цифрой.
\end{enumerate}

\textbf{Пример:} Переведём \(1011101_2\) в восьмеричную систему.
\[
1011101_2 \quad \Rightarrow \quad \underline{1}\, 011\, 101.
\]
Добавим ведущие нули для первой группы: \(001\), получим группы: \(001\), \(011\), \(101\).
\begin{itemize}
    \item \(001_2 = 1_8\),
    \item \(011_2 = 3_8\),
    \item \(101_2 = 5_8\).
\end{itemize}
Таким образом, \(1011101_2 = 135_8\).

\subsection{Двоичная \(\leftrightarrow\) Шестнадцатеричная}
Поскольку \(16 = 2^4\), процесс схож с преобразованием в восьмеричную:
\begin{enumerate}
    \item Разбиваем двоичное число на группы по 4 бита (начиная с младших разрядов).
    \item Каждую группу заменяем соответствующей шестнадцатеричной цифрой.
\end{enumerate}

\textbf{Пример:} Переведём \(10101100_2\) в шестнадцатеричную систему.
\[
10101100_2 \quad \Rightarrow \quad 1010 \quad 1100.
\]
\begin{itemize}
    \item \(1010_2 = A_{16}\) (так как \(10_{10}=A\)),
    \item \(1100_2 = C_{16}\) (так как \(12_{10}=C\)).
\end{itemize}
Итак, \(10101100_2 = AC_{16}\).

\subsection{Шестнадцатеричная \(\leftrightarrow\) Восьмеричная}
Преобразование между шестнадцатеричной и восьмеричной системами часто выполняется через десятичную или двоичную систему. Один из удобных способов:
\begin{enumerate}
    \item Перевести шестнадцатеричное число в двоичное (заменяя каждую шестнадцатеричную цифру на 4 двоичных бита).
    \item Полученное двоичное число преобразовать в восьмеричное, группируя биты по 3.
\end{enumerate}

%%%%%%%%%%%%%%%%%%%%%%%%%%%%%%%%%%%%%%%%%%%%%%%%%%%%%%%%%%%%%%
\section{Разбор задач с подробным решением}

\subsection{Задача 1. Перевести число из двоичной системы в шестнадцатеричную}
\textbf{Условие:} Переведите число \(11010110_2\) в шестнадцатеричную систему.

\textbf{Решение:}
\begin{enumerate}
    \item Разобьём число на группы по 4 бита, начиная с младших разрядов:
    \[
    11010110_2 \quad \Rightarrow \quad 1101 \quad 0110.
    \]
    \item Преобразуем каждую группу:
    \begin{itemize}
        \item \(1101_2 = 1\cdot2^3+1\cdot2^2+0\cdot2^1+1\cdot2^0 = 8+4+0+1 = 13 \Rightarrow D_{16}\),
        \item \(0110_2 = 0\cdot2^3+1\cdot2^2+1\cdot2^1+0\cdot2^0 = 0+4+2+0 = 6 \Rightarrow 6_{16}\).
    \end{itemize}
    \item Таким образом, \(11010110_2 = D6_{16}\).
\end{enumerate}

\subsection{Задача 2. Перевести число из шестнадцатеричной системы в двоичную}
\textbf{Условие:} Переведите число \(2F4_{16}\) в двоичную систему.

\textbf{Решение:}
\begin{enumerate}
    \item Преобразуем каждую шестнадцатеричную цифру в 4-битное двоичное число:
    \begin{itemize}
        \item \(2_{16} = 0010_2\),
        \item \(F_{16} = 15_{10} = 1111_2\),
        \item \(4_{16} = 0100_2\).
    \end{itemize}
    \item Объединяем группы:
    \[
    2F4_{16} = 0010\,1111\,0100_2.
    \]
    \item Итоговое двоичное число: \(001011110100_2\) (лидирующие нули можно опустить: \(1011110100_2\)).
\end{enumerate}

\subsection{Задача 3. Перевести число из восьмеричной системы в десятичную}
\textbf{Условие:} Переведите число \(657_8\) в десятичную систему.

\textbf{Решение:}
\[
657_8 = 6\cdot8^2 + 5\cdot8^1 + 7\cdot8^0 = 6\cdot64 + 5\cdot8 + 7 = 384 + 40 + 7 = 431_{10}.
\]

\subsection{Задача 4. Перевести число из десятичной системы в двоичную}
\textbf{Условие:} Переведите число \(237_{10}\) в двоичную систему.

\textbf{Решение:} Выполним последовательное деление на 2:
\begin{enumerate}
    \item \(237 \div 2 = 118\) с остатком \(1\).
    \item \(118 \div 2 = 59\) с остатком \(0\).
    \item \(59 \div 2 = 29\) с остатком \(1\).
    \item \(29 \div 2 = 14\) с остатком \(1\).
    \item \(14 \div 2 = 7\) с остатком \(0\).
    \item \(7 \div 2 = 3\) с остатком \(1\).
    \item \(3 \div 2 = 1\) с остатком \(1\).
    \item \(1 \div 2 = 0\) с остатком \(1\).
\end{enumerate}
Остатки в обратном порядке: \(1\,1\,1\,0\,1\,1\,0\,1\). Таким образом,
\[
237_{10} = 11101101_2.
\]

\subsection{Задача 5. Перевести число из двоичной системы в восьмеричную}
\textbf{Условие:} Переведите число \(101110111_2\) в восьмеричную систему.

\textbf{Решение:}
\begin{enumerate}
    \item Разобьём число на группы по 3 бита, начиная с правого края:
    \[
    101110111_2 \quad \Rightarrow \quad 101\,110\,111.
    \]
    \item Преобразуем каждую группу:
    \begin{itemize}
        \item \(101_2 = 5_8\),
        \item \(110_2 = 6_8\),
        \item \(111_2 = 7_8\).
    \end{itemize}
    \item Итог: \(101110111_2 = 567_8\).
\end{enumerate}

\subsection{Задача 6. Перевести число из шестнадцатеричной системы в восьмеричную}
\textbf{Условие:} Переведите число \(3B_{16}\) в восьмеричную систему.

\textbf{Решение:}
\begin{enumerate}
    \item Сначала переведём шестнадцатеричное число в двоичное. Каждую цифру заменим на 4 бита:
    \begin{itemize}
        \item \(3_{16} = 0011_2\),
        \item \(B_{16} = 11_{10} = 1011_2\) (при необходимости дополним нулями: \(1011_2\)).
    \end{itemize}
    \item Объединяем: \(3B_{16} = 0011\,1011_2\).
    \item Разобьём двоичное число на группы по 3 бита, начиная с правого края:
    \[
    0011\,1011_2 \quad \Rightarrow \quad 00\, 111\, 011.
    \]
    Добавим недостающие нули к первой группе: \(000\).
    \item Преобразуем:
    \begin{itemize}
        \item \(000_2 = 0_8\),
        \item \(111_2 = 7_8\),
        \item \(011_2 = 3_8\).
    \end{itemize}
    \item Таким образом, \(3B_{16} = 073_8\) (лидирующий 0 можно опустить: \(73_8\)).
\end{enumerate}

%%%%%%%%%%%%%%%%%%%%%%%%%%%%%%%%%%%%%%%%%%%%%%%%%%%%%%%%%%%%%%
\section{Особенности и советы при работе с системами счисления}
\begin{itemize}
    \item При переводе из двоичной системы в восьмеричную или шестнадцатеричную важно правильно группировать биты (по 3 для восьмеричной и по 4 для шестнадцатеричной). Если общее число бит не кратно 3 или 4, дополняйте число ведущими нулями.
    \item Для проверки правильности перевода можно выполнить обратное преобразование.
    \item В системах с основанием, большим 10, следует помнить соответствие цифр и букв (например, \(A=10,\, B=11,\, \dots, F=15\)).
\end{itemize}

%%%%%%%%%%%%%%%%%%%%%%%%%%%%%%%%%%%%%%%%%%%%%%%%%%%%%%%%%%%%%%
\section{Заключение}
В данной работе рассмотрены основы работы с различными системами счисления, а также приведены методы преобразования чисел между бинарной, восьмеричной, десятичной и шестнадцатеричной системами. Подробно разобраны примеры, позволяющие на практике закрепить теоретические знания. Освоение этих методов является важным навыком для студентов, изучающих информатику и программирование, поскольку различные системы счисления активно применяются в вычислительной технике и алгоритмах.

%%%%%%%%%%%%%%%%%%%%%%%%%%%%%%%%%%%%%%%%%%%%%%%%%%%%%%%%%%%%%%
\section*{Список использованных источников}
\addcontentsline{toc}{section}{Список использованных источников}
\begin{enumerate}
    \item Куликов А.В. <<Основы дискретной математики>>.
    \item Левин А.<<Информатика: теория и практика>>.
    \item Методические материалы кафедры информатики.
\end{enumerate}

%%%%%%%%%%%%%%%%%%%%%%%%%%%%%%%%%%%%%%%%%%%%%%%%%%%%%%%%%%%%%%
\section*{Приложение. Дополнительные примеры}
\addcontentsline{toc}{section}{Приложение}

\textbf{Пример A.} Перевод числа \(1001110_2\) в шестнадцатеричную систему.
\begin{enumerate}
    \item Группировка по 4 бита: \(0100\ 1110_2\).
    \item \(0100_2 = 4_{16}\), \(1110_2 = E_{16}\).
    \item Ответ: \(1001110_2 = 4E_{16}\).
\end{enumerate}

\textbf{Пример B.} Перевод числа \(572_{10}\) в восьмеричную систему.
\begin{enumerate}
    \item Деление на 8:
    \begin{itemize}
        \item \(572 \div 8 = 71\) с остатком \(4\).
        \item \(71 \div 8 = 8\) с остатком \(7\).
        \item \(8 \div 8 = 1\) с остатком \(0\).
        \item \(1 \div 8 = 0\) с остатком \(1\).
    \end{itemize}
    \item Остатки в обратном порядке: \(1\,0\,7\,4\).
    \item Ответ: \(572_{10} = 1074_8\).
\end{enumerate}

\textbf{Пример C.} Перевод числа \(3C9_{16}\) в десятичную систему.
\[
3C9_{16} = 3\cdot16^2 + 12\cdot16^1 + 9\cdot16^0 = 3\cdot256 + 192 + 9 = 768 + 192 + 9 = 969_{10}.
\]

\vfill
\textbf{Примечание:} Приведённый материал рассчитан на подробное изучение темы и может служить как справочное пособие для студентов, изучающих основы систем счисления.

\end{document}
